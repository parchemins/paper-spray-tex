
\begin{abstract}
  Deploying peer sampling services on top of WebRTC allows running large scale
  distributed applications on a network of browsers. However, WebRTC has
  several constraints that makes existing peer sampling services inefficient or
  unreliable.  In this paper, we propose \SPRAY, an adaptive
  peer-sampling service designed to meet the WebRTC constraints. \SPRAY
  dynamically adapts the neighborhood of each peer to the network
  size. Furthermore, it only uses neighbor-to-neighbor interactions to
  establish the connections and converges exponentially fast to a topology
  exposing properties similar to random graphs. Experiments demonstrate the
  adaptiveness of \SPRAY and highlight its efficiency improvement compared
  to \CYCLON at the cost of a small overhead.
\end{abstract}

\category{H.4}{TODO}{EXAMPLE BELOW}
\category{D.2.8}{Software Engineering}{Metrics}[complexity measures, performance measures]

\terms{Theory, example of term}

\keywords{Random peer sampling, network overlay, adaptive}

%%% Local Variables:
%%% mode: latex
%%% TeX-master: "../paper"
%%% End:
