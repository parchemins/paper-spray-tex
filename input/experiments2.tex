
\subsection{Experimentation}
\label{subsec:experiments2}

We highlight the improvements brought to protocols built on top of \SPRAY on a
real-life use case about decentralized collaborative editing in web browsers.

% These experiments run on the Grid'5000 testbed and involve up to 600 browsers
% simultaneously writing a document. Using \SPRAY, we expect a traffic
% logarithmically scaling compared to the network size, along with a
% polylogarithmic growth brought by the real-time editor itself.

%%\vspace{-7pt}
%%\paragraph{CRATE on network of browsers}

\begin{asparadesc}
\item [Objective:] To show the influence of \SPRAY's adaptiveness over the
  traffic generated by decentralized editors running on a network of browsers.
\item [Description:] Experiments run on Grid'5000 where machines host 5 browsers
  each. Browsers open \CRATE and connect to an editing session through a
  signaling server.  Runs comprise from 101 browsers to 601 browsers with 100
  browsers increments, i.e., 6 different runs.  The first editor creates the
  editing session which is progressively joined by the other writers (1 joiner
  per 5 seconds). Each member starts sharing the access to the editing session
  as soon as it joins it. Outsiders join the network through one of them chosen
  at random. Once all peers have joined the editing session, the document
  editing starts. The insertion rate is 100 insertions per second uniformly
  distributed among peers. Each experiment runs during 8 hours of which 7 hours
  are dedicated to editing. The document size reaches millions of characters.

\begin{figure}
  \centering
  \includegraphics[width=\SCALE\textwidth]{img/traffic.eps}
  \caption{\label{fig:traffic}Average traffic per second.}
\end{figure}

\item [Results:] Figure~\ref{fig:traffic} shows the average traffic
  per second of members involved in the editing session. The x-axis
  denotes the time progression of the experiment in percentage over
  the 7 hours. The y-axis denotes the traffic generated by each
  browser in megabytes per second, i.e., 100 operations are
  broadcast. The legend shows the average partial view and the
  variance associated with each run. The height of the plots
  corresponds to the multiplicative factor coming from the messages
  dissemination. As expected, this factor grows logarithmically
  regarding the network size. Thus, 101 browsers have an average
  traffic lower than 601 browsers because their partial views are
  smaller in average.  On the opposite, using \CYCLON, the traffic
  would have been the same for all runs. Since it commonly
  overestimates partial views to accommodate with any network size,
  the traffic would have been higher (cf. Figure~\ref{fig:churn}). It
  is important since even small partial view size differences
  significatively impact traffic.  Figure~\ref{fig:traffic} also shows
  that the average partial view size follows the natural logarithmic
  expectation. Yet, the run involving 501 browsers has a slightly
  higher average partial view size than the run involving 601
  browsers. Because the joining part of \SPRAY establishes a number of
  WebRTC connections depending on the first contact member, there are
  variations between independant runs. Still, \SPRAY scales
  logarithmically overall. Figure~\ref{fig:traffic} finally shows that
  the variance of partial views $\sigma^2$---displayed in the
  legend---stays small, which indicates that the network reached a
  state where neighborhood sizes are balanced, hence, where the load is
  balanced.
\item [Reasons:] Broadcast uses neighborhoods to disseminate messages. Each
  member receives and forwards each operation which transitively reaches all
  members. Thus, the traffic depends on messages size multiplied by
  neighborhoods size logarithmically scaling thanks to \SPRAY. The growth during
  each run corresponds to the polylogarithmic growth of identifiers from the
  editors. Since the document size increases over time, the \LSEQ's identifiers
  grow accordingly which impacts on messages size~\cite{nedelec2013lseq}.
\end{asparadesc}




%%% Local Variables:
%%% mode: latex
%%% TeX-master: "../paper"
%%% End:
