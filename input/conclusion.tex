
\section{Conclusion and perspectives}
\label{sec:conclusion}

In this paper, we introduced a random peer sampling approach called Scamplon.
This new approach, inspired of two state-of-the-art approaches, namely Scamp
and Cyclon, provide the best of both worlds:
\begin{inparaenum}
\item logarithmically growing partial views compared to the global network
  size,
\item a quick convergence to a random graph
\item using neighbour-to-neighbour interactions.
\end{inparaenum}
As such, Scamplon outperforms its parents in the context of 1-way connection
establishments. In addition, it greatly outperforms state-of-the-art in the
3-way handshake connection establishment set-up which becomes increasingly
important with recent technologies such as WebRTC that allows peer-to-peer
connection within modern browser. Since, random peer sampling constitutes the
ground of many decentralised applications.

Future work includes an analysis of the chosen size of exchanges in
Scamplon. Indeed, currently, each peer choose a linear number of neighbours to
exchange compared to the size of partial view:
$\left\lceil |\mathcal{P}|\over{2} \right\rceil$. Nevertheless, the paper
introducing Cyclon shows that a small subset of the partial view is enough to
provide a nearly optimal convergence speed.

Future work also includes a Javascript implementation using WebRTC. While
experimentation of this paper shows the results of Peersim simulations, the
real implementation would open the gate to emulations, and even real
peer-to-peer decentralised applications (MOAR).

%%% Local Variables:
%%% mode: latex
%%% TeX-master: "../paper"
%%% End:
