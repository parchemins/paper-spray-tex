
\section{Conclusion and perspectives}
\label{sec:conclusion}

In this paper, we introduced a random peer sampling approach called
\SPRAY.  This new approach, inspired of two state-of-the-art approaches,
namely \SCAMP and \CYCLON, provide the best of both worlds:
\begin{inparaenum}[(i)]
\item logarithmically growing partial views reflecting the global network size,
\item constant time complexity on connection establishments using solely
  neighbor-to-neighbor interactions,
\item exponentially fast convergence to a random graph.
\end{inparaenum}
As such, \SPRAY constitutes an improvement over state-of-the-art
approaches in the traditional context of 1-way connection establishments. In
addition, the improvement becomes crucial in the 3-way handshake connection
set-up which is increasingly important since recent technologies (e.g. WebRTC)
allow peer-to-peer connections between modern web browsers. Random peer
sampling protocols constitute the ground of many decentralized applications.
Thus, we expect that decentralized applications (requiring scalable broadcast
among other) will flourish in the web browsers, while they were previously
quartered to standalone applications.

Future work includes investigations on topology managers such as
T-Man~\cite{jelasity2009tman} or Vicinity~\cite{voulgaris2005epidemic}. Indeed,
they traditionally rely on random peer sampling approaches using fixed-size
partial view. Thus, they maintain a fixed-size view of their most closely
related neighbors using a ranking function. With \SPRAY, it is possible to
extend their behaviour to use dynamic partial views. If the view size could
adapt to the size of a cluster (if the topology creates disjoint clusters), it
would improve the traffic, robustness, etc. Future work also includes a
Javascript implementation using WebRTC. While experimentation of this paper
shows the results of Peersim simulations, the real implementation would open
the gate to emulations, and even real peer-to-peer distributed and
decentralized applications.

%%% Local Variables:
%%% mode: latex
%%% TeX-master: "../paper"
%%% End:
