
\section{Introduction}
Random peer sampling (REF) is an essential ground to many distributed
applications. Since the peers cannot afford to have global knowledge of the
network membership, they only have a partial view of the latter. Using these
partial views, the peers are able to perform many tasks. In particular, it
allows a scalable yet highly reliable broadcasting which constitutes the core
of numerous decentralized applications (REFs). However, current random peer
sampling approaches are not entirely satisfying. 

The Cyclon protocol~\cite{voulgaris2005cyclon} provides each peer with a fixed
size partial view.  Each element in this view has an age which allows quick
withdrawing of left peers. Nevertheless, the user must foresee the maximum size
of the network to set the appropriate view size. As such, the views are
generally oversized. On the other hand, the Scamp
protocol~\cite{ganesh2003peer} incrementally builds the partial views of peers
at join. Thus, each peer has a partial view size logarithmically growing
compared to the global size of the network. Nevertheless, a peer may connect to
peers at several hops from him. Hence, the connections are more likely to fail.
Furthermore, both approaches do not consider three-way handshake connections
set-up which becomes increasingly imporant with technologies such as WebRTC,
allowing peer-to-peer connections establishment directly within modern web
browser. In such context, each connection matters. Therefore, oversizing the
partial views as in Cyclon becomes a heavy burden. Also, handshaking requires a
round trip of joining messages which increases the Scamp's complexity of
failures. Scamp becomes unusable in pratice.

In this paper, we introduce Scamplon, a random peer sampling protocol
\begin{inparaenum}[(i)]
\item which provides each peer with a logarithmically growing partial view size
  compared to the global network size,
\item which quickly converges to a random graph,
\item which can handle three-way handshake.
\end{inparaenum}
Experiments shows that it outperforms the state-of-the-art approaches both in
one-way connections and three-way handshake connections.

The rest of this paper is organized as follow: Section~\ref{sec:background}
introduces the necessary background to understand Scamplon and highlights our
motivations. Section~\ref{sec:proposal} details the Scamplon protocol.
Section~\ref{sec:experiments} shows the properties of Scamplon and compare them
to state-of-the-art random peer sampling
approaches. Section~\ref{sec:relatedwork} reviews the related work. We conclude
and discuss about perspective in Section~\ref{sec:conclusion}.

%%% Local Variables:
%%% mode: latex
%%% TeX-master: "../paper"
%%% End:
