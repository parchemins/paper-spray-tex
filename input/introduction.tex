
\section{Introduction}

Peer-sampling
protocols~\cite{jelasity2007gossip,tolgyeski2009adaptive,voulgaris2005cyclon}
constitute a fundamental mechanism for a number of large-scale distributed
applications both on the Cloud~\cite{decandia2007dynamo} and in a peer-to-peer
setting~\cite{Frey09Middleware,voulgaris2005sub,wuhib2009robust}. By providing
each node with a continuously changing partial view of the network, they make
applications resilient to churn~\cite{bertier-d2ht} and inherently load
balancing~\cite{Frey09DSN}. In the context of video streaming, for example, a
peer-sampling protocol makes it possible to distribute the streaming load over
all peers without requiring the creation and the maintenance of rigid structures
like multiple trees~\cite{Frey09DSN,monod:THESIS}.

The recent introduction of WebRTC~\cite{webrtc} has renewed the
research interest in a variety of applications that require
peer-sampling protocols such as video streaming~\cite{hivejs,smoothcache2},
content-delivery networks~\cite{Zhang:2013:MBC:2465351.2465379}, or
real-time collaborative editors~\cite{nedelec2016crate}. However,
deploying existing peer-sampling protocols on top of WebRTC raises
important technical challenges.
\begin{inparaenum}[(1)]
\item WebRTC does not manage addressing nor routing; this makes
  connection establishment much more costly than on IP networks and
  more likely to fail. 
\item Browsers run on desktops, laptops and mobile phones. This
  requires protocols that reduce resource consumption as much as
  possible.
\item The ability to launch WebRTC sessions through simple HTTP links
  exposes applications to sudden bursts of popularity.
\end{inparaenum}
Consider the example of a user who is streaming a video directly from
his mobile phone to some of his friends. The user suddenly witnesses
some dramatic event, and his friends spread the news by twitting the
stream's address. Instantly a huge number of users connect and start
watching the stream on their laptops and phones. The streaming
mechanisms and the protocols it relies on must be able to adapt to
this sudden burst of popularity, maintaining their quality of service,
while being able to return to their initial configuration when the
popularity burst subsides. 

Unfortunately, existing peer-sampling protocols lack adaptiveness or
reliability in the context of WebRTC. On the one hand,
\SCAMP~\cite{ganesh2001scamp,ganesh2003peer} provides adaptiveness by
maintaining partial views of size $\ln(n)+k$, $n$ being the number of
nodes and $k$ being a constant. However, \SCAMP is not reliable in the
context of WebRTC, for its connection establishment process is based
on random walks and cannot handle the connection failures that often
occur in WebRTC. On the other hand, the most popular approaches like
\CYCLON~\cite{voulgaris2005cyclon} and the whole RPS protocol
family~\cite{jelasity2007gossip} are reliable in the context of WebRTC
but are not adaptive: developers have to configure fixed-size views at
deployment time. This forces developers to oversize partial views to
handle potential bursts, either wasting resources or not provisioning
for large-enough settings. A simple solution to address this problem
estimates the actual number of network nodes by periodically running
an aggregation protocol~\cite{montresor2004robust}, then resizes the
partial views. However, this aggregation protocol would have to run
quite frequently, resulting in significant network overhead to
anticipate a popularity burst that may never happen. Our approach
provides adaptiveness and reliability by locally self-adjusting the
size of partial views at join, leave, and shuffle times. At a marginal
cost, partial view size converges towards $\ln(n)$. This not only
makes our approach adaptive but also outputs the size of the network
to the application level allowing applications to adapt to sudden
burst in popularity at no additional cost. We demonstrate the impact
of our approach on two use-cases that use RPS to broadcast
messages. The first one implements a live video streaming based on
three-phase gossip~\cite{FlightPath,Frey09DSN,monod:THESIS}. In this
case, adaptiveness allows message delivery to remain stable even in
presence of burst in popularity. The second one implements a real-time
collaborative editor. In this case, adaptiveness allows the editors to
adjust the traffic to the size of the network. Some distributed hash
table protocols such as D2HT~\cite{bertier-d2ht} require both the
knowledge of the size of the network to compute the view of each node,
and a random peer sampling protocol to provide connectivity in the
presence of high churn. Our approach provides both at the same cost.

% The scientific challenge consists in finding a way to adapt to the
% network size without actually measuring its number of members as
% proposed by Scamp, but running on top of WebRTC.

In this paper, we address the challenge of a dynamically adaptive peer-sampling,
by introducing \SPRAY, a novel random peer-sampling protocol inspired by both
\SCAMP~\cite{ganesh2001scamp,ganesh2003peer} and
\CYCLON~\cite{voulgaris2005cyclon}. \SPRAY improves the state-of-the-art in
several ways.
\begin{inparaenum}[(i)]
\item It dynamically adapts the neighborhood of each peer. Thus, the number of
  connections scales logarithmically with the network size.
\item It only uses neighbor-to-neighbor interactions to establish
  connections. Thus, a node needs to rely only on a constant number of
  other nodes to establish a connection regardless of network size.
\item It quickly converges to a topology with properties similar to
  those of random graphs. Thus, the network becomes robust to massive
  failures, it can quickly disseminate messages.
\item Experiments show that \SPRAY provides adaptiveness and reliability at a
  marginal cost.
\item Use-cases demonstrate how the estimation of the network size positively
  impacts two RPS-based broadcast protocols either on traffic or message
  delivery.
\end{inparaenum}

The rest of this paper is organized as follows: Section~\ref{sec:relatedwork}
reviews the related work. Section~\ref{sec:problem} states the scientific
problem. Section~\ref{sec:proposal} details the \SPRAY
protocol. Section~\ref{sec:experimentation} presents experimentation results of
\SPRAY and compares them to state-of-the-art. Section~\ref{sec:usecase} details
our use-cases. We conclude in Section~\ref{sec:conclusion}.




%%% Local Variables:
%%% mode: latex
%%% TeX-master: "../paper"
%%% End:
