
\section{Introduction}

Peer sampling service is a fundamental mechanism for many large-scale
distributed applications including information
dissemination~\cite{eugster2003lightweight, tolgyeski2009adaptive},
aggregation~\cite{jelasity2004epidemic} and network
management~\cite{jelasity2009tman, voulgaris2005epidemic}. Recently,
WebRTC\footnote{\url{http://www.webrtc.org/}} opened the opportunity to deploy
large scale distributed applications on browsers that can run on laptops,
desktops and mobile devices. Among others, WebRTC drastically simplifies the
deployment of large-scale distributed applications even within complex network
systems that utilize firewalls, proxies and Net Address Translation (NAT).

Unfortunately, WebRTC has several constraints that make existing peer sampling
services inefficient or unreliable. Brow\-sers can run on small devices in
mobile networks. Hence, keeping the number of connections as low as possible is
a major requirement. However, peer sampling services such as
\CYCLON~\cite{voulgaris2005cyclon} do not adapt the number of connections to
the real number of participants. For instance, a user must maintain 10
connections with other remote browsers when only 6 are enough. On the other
hand, the peer sampling service \SCAMP~\cite{ganesh2003peer} is adaptive but
uses random dissemination paths to establish connections which is much more
costly and likely to fail in WebRTC context.

In this paper, we introduce \SPRAY, a random peer sampling protocol inspired by
both \SCAMP~\cite{ganesh2003peer}
and \CYCLON~\cite{voulgaris2005cyclon}. Compared to the state of art,
\begin{inparaenum}[(i)]
\item \SPRAY dynamically adapts the neighborhood of each peer. Thus, the number
  of connections grows logarithmically compared to the size of the network.
\item \SPRAY only uses neighbor-to-neighbor interactions to establish
  connections. Thus, the connections are established in constant time.
\item \SPRAY quickly converges to a topology exposing properties similar to
  those of random graphs. Thus, the network becomes robust to massive failures,
  efficiently disseminates information etc.
\item In the experimental setup, we show the adaptiveness of \SPRAY and
  highlight its efficiency improvement compared to \CYCLON and \SCAMP, at the
  cost of little overhead.
\end{inparaenum}

The rest of this paper is organized as follows: Section~\ref{sec:relatedwork}
reviews the related work. Section~\ref{sec:proposal} details the \SPRAY
protocol. Section~\ref{sec:experiments} shows the properties of \SPRAY
and compares them to state-of-the-art random peer sampling approaches. We
conclude and discuss about the perspective in Section~\ref{sec:conclusion}.

%%% Local Variables:
%%% mode: latex
%%% TeX-master: "../paper"
%%% End:
